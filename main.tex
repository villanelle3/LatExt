\documentclass[12pt,a4paper]{article}
%%%%%%%%%%%%%%%%%%%%%%%%%%%%%%%%%%%%%%%%%%%%%%%%%%%%%
%\usepackage[utf8]{inputenc}
\usepackage[portuguese]{babel}
\addto{\captionsportuguese}{%
  \renewcommand{\refname}{6  \hspace{0.3cm}  REFERÊNCIAS BIBLIOGRÁFICAS}
}
\addto{\captionsportuguese}{% 
  \renewcommand{\contentsname}%
    {\centering Sumário}%
}
\usepackage[T1]{fontenc}
\usepackage{fontspec}
\usepackage{amsmath,esint}
\usepackage{amsfonts}
\usepackage{amssymb}
\usepackage{makeidx}
\usepackage{graphicx}
\usepackage{indentfirst}
\usepackage{lettrine}
\usepackage{lmodern}
\usepackage{dirtytalk}
\usepackage{setspace}
\usepackage{pdfpages}
\usepackage[left=3cm,right=2cm,top=3cm,bottom=2cm]{geometry}
\usepackage{hyperref}
\usepackage{multirow}
\usepackage{booktabs}
\usepackage{subfigure}
\usepackage{abntex2cite}
\usepackage{fancyhdr}
%-----------------------------------------------------------
\hypersetup{
%links coloridos
   pdfborder={0 0 0},
   colorlinks = true,  
   linkcolor = black,          % Cor dos links internos
   citecolor = blue        % Cor dos links das referências bibliográficas
}
\setmainfont{Times New Roman}
\pagestyle{fancy}
% Clear the header and footer
\fancyhead{}
\fancyfoot{}
\fancyhead[R]{\thepage}
\renewcommand{\headrulewidth}{0pt}%
\setlength{\headheight}{14.49998pt}
%-----------------------------------------------------------

\begin{document}

%___________________________CAPA___________________________%
    \begin{titlepage}

        \begin{center}
            \begin{Large}
                \textbf{UNIVERSIDADE FEDERAL DE GOIÁS}
            
                \textbf{INSTITUTO DE FÍSICA} 
            \end{Large}
        \end{center}
   
   
    \vspace*{5cm}

    \begin{center}
      \begin{Large}
         \textbf{Relatório Final de Estágio\\
         \vspace{0.5cm}
         <ano/período>}
      \end{Large}
       
    \end{center}
      
    \vspace{5cm}  
      
    \begin{center}
    \begin{Large}
        \textbf{<nome \\
        \vspace{0.5cm}
        matrícula>}
    \end{Large}
       
    \end{center}
    
    \vspace{4cm}
    
    \begin{center}
    \begin{large}
        \textbf{local, data}
    \end{large}
        
    \end{center}
    
    

    \end{titlepage}



%______________________Agradecimentos (opcional)_________________%
\begin{center}
\Large{\textbf{ Agradecimentos}} %%seção opcional
\end{center}




\newpage


%______________________________Sumário_________________________%
\tableofcontents
\newpage
%_____________________________Introdução__________________________%
\section{INTRODUÇÃO}


%_____________________________Caracterização______________________%
\section{CARACTERIZAÇÃO DA BACIA DO MUNDAÚ}

A bacia hidrográfica do rio Mundaú está localizada nos Estados de Pernambuco e Alagoas. A porção compreendida no território pernambucano (Unidade de Planejamento Hídrico UP6) localiza-se entre as coordenadas 08º 41’ 34” e 09º 14’ 00” de latitude sul, e 36º 03’36” e 36º 37’ 27” de longitude oeste.


%________________________Relatório Descritivo_____________________%
\section{RELATÓRIO DESCRITIVO}


%________________________COMPETÊNCIAS_____________________%
\section{COMPETÊNCIAS PROFISSIONAIS}


%________________________CONCLUSÃO_____________________%
\section{CONSIDERAÇÕES FINAIS}


%\bibliography{bibliografia}  %Preencha o arquivo bibliografia.bib com as referência adequadamente formatadas (pesquise sobre arquivos de extensão .bib) e descomente esta linha para que aparecem já em formato ABNT

\end{document}
